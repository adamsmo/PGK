\documentclass{article}
\usepackage{graphicx}
\usepackage[polish]{babel}
\begin{document}

\title{Sprawozdanie z projektu\\ Grafika komputerowa}
\author{Adam Smolarek}

\maketitle

\section{Opis zagadnienia}
Przedmiotem projektu jest zrealizowanie renderowania 3d bez użycia akceleracji sprzętowej. Jeżeli chodzi o samą modelowaną scene to będzie się ona składała z prostopadłościanów. Po zamodelowanej przestrzeni będzie można się poruszać w 3 kierunkach oraz wykonywać obroty w wokół 3 osi kartezjańskiego układu współżędnych. Poza tym dostępna będzie rónież opcja zoom polegająca na skalowaniu otrzymanego obrazu. Wymienione operacje będą kontrolowane przy pomocy klaiwatury (poruszanie się oraz obroty) natomiast sam zoom będzie kontrolowany przez kółko myszki. 

Do zrealizowania projektu użyję języka Java oraz standardowej biblioteki do rysowania interfejsu użytkownika - AWT. Jeżeli chodzi o samo rendrowanie to będzie ono zrealizowane przez żutowanie siatki obiektu w przestrzeni 3D na płąszczyznę (nieskończoną) reprezentującą ekran. Następnie uzyskane w ten sposób odcinki będą przycinane do granic ekranu i wyświetlane przy użyciu biblioteki AWT na oknie głównym. Animacja będzie realizowana tylko przy interakcji użytkownika z programem, w przypadku braku interakcji będzie wyświetlany statyczny obraz reprezentujący ostatnią wygenerowaną klatkę.

\subsection{Użyte przekształcenia}
Macież translacji

\[
Mt = \begin{bmatrix}
       1 & 0 & 0 & 0 \\[0.3em]
       0 & 1 & 0 & 0 \\[0.3em]
       0 & 0 & 1 & 0
       X & Y & Z & 1
     \end{bmatrix}
\]


Macież rotacji
\[
 Mx = \begin{bmatrix}
       1 & 0 & 0 & 0 \\[0.3em]
       0 & cos(a) & sin(a) & 0 \\[0.3em]
       0 & sin(a) & cos(a) & 0
       X & Y & Z & 1
     \end{bmatrix}
\]

\[
 My = \begin{bmatrix}
       cos(a) & 0 & sin(a) & 0 \\[0.3em]
       0 & 1 & 0 & 0 \\[0.3em]
       sin(a) & 0 & cos(a) & 0
       0 & 0 & 0 & 1
     \end{bmatrix}
\]

\[
 Mz = \begin{bmatrix}
       cos(a) & sin(a) & 0 & 0 \\[0.3em]
       sin(a) & cos(a) & 0 & 0 \\[0.3em]
       0 & 0 & 1 & 0
       0 & 0 & 0 & 1
     \end{bmatrix}
\]




\section{Plan pracy}
\begin{enumerate}
\item Postawieni 4 maszyn virtualnych pod kontrolą linuxa
\item Połączenie maszyn siecią
\item Zaimplemetnowanie algorytmu
\item Zainstalowanie aplikacji na komputerach
\item Testy integracyjne
\item Napisanie sprawozdania
\end{enumerate}
\end{document}
